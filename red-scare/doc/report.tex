\documentclass{tufte-handout}
\usepackage[utf8]{inputenc}
\usepackage{tikz}
\usepackage{amsmath}

\usepackage{color}
\newcommand{\red}[1]{{\color{red} #1}}
\usepackage{booktabs}
\begin{document}
\section{Red Scare! Report}

by Alice Cooper.

\subsection{Results}

The following table gives my results for all graphs of at least 500 vertices.

\medskip
\begin{tabular}{lrrrrrr}
  \toprule
  Instance name & $n$ & A & F & M & N & S \\
  \midrule
  rusty-5762 & 5,762 & true & 16 & -- & ? & 5 \\
  wall-p-10000 & 10,000 &\\	
  $\vdots$\\
  \bottomrule
\end{tabular}
\medskip

The columns are for the problems Alternate, Few, Many, None, and Some.
The table entries either give the answer, or contain `?' for those cases where I was unable to find a solution within reasonable time.
For those questions where there is a reason for my inability to find a good algorithm (because the problem is hard), I wrote `?!'.

For the complete table of all results, see the tab-separated text file {\tt results.txt}.

\subsection{Methods}
%  For problem A, I solved each instance $G$ by $\cdots$\footnote{Describe what you did.
%  Use words like ``building a inverse anti-tree without self-loops where each vertex in $G$ is presented by a Strogatz--Wasserman shtump.
%  I then performed a standard longest hash sorting using the algorithm of Bronf (Algorithm 5 in [1]).''
%  Be neat, brief, and precise.}

% TODO: Specify where BFS alg comes from
For problem A, I solved each instance $G$ by removing all edges that go between vertices of same color, unless it is $s -- t$. This new graph is then run on a shortest-path algorithm. In this case Breadth-First Search (BFS).
The running time of this algorithm is $O(V + E)$, and my implementation spends $\cdots$ seconds on the instance $\cdots$ with  $n=\cdots$.

For problem F, I solved each instance $G$ by $\cdots$
The running time of this algorithm is $\cdots$, and my implementation spends $\cdots$ seconds on the instance $\cdots$ with  $n=\cdots$.

For problem N, I solved each instance $G$ by removing all red vertices from the graph, and then feeding it to a shortest-path algorithm.
The running time of this algorithm is $O(V + E)$, and my implementation spends $\cdots$ seconds on the instance $\cdots$ with  $n=\cdots$.

I solved problem $\cdots$ for all $\cdots$\footnote{For instance, “planar, bipartite”} graphs using $\cdots$.

I was unable to solve problem $\cdots$ except for the $\cdots$ instances.
This is because, in generality, this problem is $\cdots$. 
To see this, consider the following reduction from $\cdots$.
Let $\ldots$ 

I was also unable to solve $\cdots$ for $\cdots$, but I don’t know why.\footnote{Remove or expand as necessary.}

\subsection{References}
\begin{description}
  \item[1.] \emph{APLgraphlib---A library for Basic Graph Algorithms in APL}, version 2.11, 2016, Iverson Project, {\tt github.com/iverson/APLgraphlib}.\sidenote{If you use references to code, books, or papers, be professional about it. Use whatever style you want, but be consistent.}

  \item[2.] A. Lovelace, \emph{Algorithms and Data Structures in Pascal}, Addison--Wesley 1881. 
\end{description}

\end{document}
